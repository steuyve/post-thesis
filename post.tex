\documentclass[11pt]{article}

\usepackage{hyperref}
\usepackage{mathtools}
\usepackage{amsthm}
\usepackage{amssymb}
\usepackage{MnSymbol}
\usepackage{mathrsfs}
\usepackage[arrow]{xy}
\usepackage{dsfont}
\usepackage{enumitem}
\usepackage{accents}
\usepackage{tabu}

\pagestyle{headings}

\newcommand{\Z}{\mathbb{Z}}

\theoremstyle{definition}
\newtheorem*{defn}{Definition}

\theoremstyle{definition}
\newtheorem{ex}{Example}

\theoremstyle{plain}
\newtheorem{theo}{Theorem}

\theoremstyle{plain}
\newtheorem{prop}{Proposition}

\theoremstyle{plain}
\newtheorem{lem}{Lemma}

\theoremstyle{definition}
\newtheorem{que}{Question}

\theoremstyle{definition}
\newtheorem{rem}[prop]{Remark}

\begin{document}

\author{Stephen Liu}
\title{Post-Thesis Notes}
\date{April 8, 2019}

\maketitle

\begin{prop}\label{prop:partialres}
Let $d$ be odd and $B_2^d$ be the closed $d$-dimensional Euclidean ball. Then
\begin{equation}\label{eqn:partialres}
\begin{aligned}
&\frac{d^2}{dt^2}\text{Mag}(tB_2^d)\big\vert_{t=0} = \\
&6\sum\limits_{\substack{2\leq p_2\leq m+1 \\ 0\leq q_1 \leq p_2-2 \\ 0\leq q_2 \leq m+1-p_2}}\frac{1}{2p_2-1}\left(\prod\limits_{k=2}^{q_1+2}\frac{(2k-2)}{(2k-1)}\right)\left(\prod\limits_{k=p_2+1}^{p_2+q_2}\frac{(2k-2)}{(2k-1)}\right) + \\
&6\sum\limits_{\substack{2\leq p_1\leq m-1 \\ 0\leq q_1\leq m-1-p_1}}\frac{1}{2p_1-1}\left(\prod\limits_{k=p_1+1}^{p_1+q_1}\frac{(2k-2)}{(2k-1)}\right)\left(\frac{2m}{(2m-1)(2m+1)}\right) + \\
&6\sum\limits_{\substack{2\leq p_1\leq m \\ 0\leq q_1\leq m-p_1}}\frac{1}{2p_1-1}\left(\prod\limits_{k=p_1+1}^{p_1+q_1}\frac{(2k-2)}{(2k-1)}\right)\left(\frac{1}{2m+1}\right) + \\
&6\sum\limits_{\substack{2\leq p_1\leq m \\ p_1+1\leq p_2 \leq m-1 \\ 0\leq q_1\leq p_2-p_1-1 }}\frac{1}{2p_1-1}\left(\prod\limits_{k=p_1+1}^{p_1+q_1}\frac{(2k-2)}{(2k-1)}\right)\frac{1}{2p_2-1}\left(\prod\limits_{k=p_2+1}^{m+1}\frac{(2k-2)}{(2k-1)}\right) - \\
&V_{1}\sum\limits_{\substack{1\leq p \leq m-1 \\ 0 \leq q \leq m - 1 - p}}\frac{1}{2p+1}\prod\limits_{k=p+1}^{p+q}\left(\frac{2k}{2k+1}\right).
\end{aligned}
\end{equation}
\end{prop}

\subsection*{Skip Factorials}

This section describes ongoing work to further simplify the expression arrived at in Proposition \ref{prop:partialres}. The products
\begin{align}
&\prod\limits_{k=a}^{b}\frac{(2k-2)}{(2k-1)}, \label{eqn:Nskipfact}\\
&\prod\limits_{k=a}^b\frac{2k}{2k+1} \label{eqn:Dskipfact}
\end{align}
that appear in the sums in ({\ref{eqn:partialres}) above are ratios of skip or double factorials. In the following, we will use the identities about skip factorials given below \cite{weisstein_double_nodate}:
\begin{align*}
&(2n)!! = 2^nn!, \\
&(2n-1)!! = \frac{(2n)!}{2^nn!}, \\
&(2n+1)!! = \frac{(2n+1)!}{2^nn!}.
\end{align*}
We will also use Catalan numbers \cite{weisstein_catalan_nodate}, which are defined by
\begin{equation*}
C_n = \frac{1}{n+1}\binom{(2n)}{n} = \frac{1}{2n+1}\binom{(2n+1)}{n}.
\end{equation*}
Using these identities, we rewrite the product (\ref{eqn:Nskipfact}):
\begin{equation}\label{eqn:Nskipfactsimp}
\begin{aligned}
\prod\limits_{k=a}^{b}\frac{(2k-2)}{(2k-1)} &= \frac{2(a-1)}{2a-1}\cdots\frac{2(b-1)}{2b-1} = \frac{(2(b-1))!!}{(2(a-2))!!}\left[\frac{(2b-1)!!}{(2(a-1)-1)!!}\right]^{-1} \\
&= \frac{2^{b-1}(b-1)!}{2^{a-2}(a-2)!}\left[\frac{(2b)!}{2^bb!}\frac{2^{a-1}(a-1)!}{(2(a-1))!}\right]^{-1} \\
&= \frac{2^{2b-1}}{2^{2(a-1)-1}}\frac{b!(b-1)!}{(2b)!}\frac{(2(a-1))!}{(a-1)!(a-2)!} = \frac{2^{2b-1}}{2^{2(a-1)-1}}\frac{(a-1)\binom{2(a-1)}{a-1}}{b\binom{2b}{b}} \\
&= \frac{2^{2b-1}}{2^{2(a-1)-1}}\frac{(a-1)aC_{a-1}}{b(b+1)C_b}.
\end{aligned}
\end{equation}
And similarly we rewrite (\ref{eqn:Dskipfact}):
\begin{equation}\label{eqn:Dskipfactsimp}
\begin{aligned}
\prod\limits_{k=a}^b\frac{2k}{2k+1} &= \frac{2a}{2a+1}\cdots\frac{2b}{2b+1} = \frac{(2b)!!}{(2(a-1))!!}\left[\frac{(2b+1)!!}{(2a-1)!!}\right]^{-1} \\
&= \frac{2^bb!}{2^{a-1}(a-1)!}\left[\frac{(2b+1)!}{2^bb!}\frac{2^aa!}{(2a)!}\right]^{-1} = \frac{2^{2b}}{2^{2a-1}}\frac{(b!)^2}{(2b+1)!}\frac{(2a)!}{a!(a-1)!} \\
&= \frac{2^{2b}}{2^{2a-1}}\frac{a\binom{2a}{a}}{(b+1)\binom{2b+1}{b}} = \frac{2^{2b}}{2^{2a-1}}\frac{a(a+1)C_a}{(2b+1)(b+1)C_b}.
\end{aligned}
\end{equation}

\begin{rem}
Here is our general approach to simplifying sums in (\ref{eqn:partialres}):
\begin{enumerate}[label=\arabic*.]
\item Use (\ref{eqn:Nskipfactsimp}) or (\ref{eqn:Dskipfactsimp}) to simplify the products appearing in the sum.
\item Employ the change of variables $k = p+q$ and summing over $k$ and $p$ to rewrite the result of step 1 above into a sum over $k$ multiplied by a sum over $p$ (where the top bound is dependent on $k$).
\item Use Wolfram Alpha to get a simple expression for the inner sum (the sum over $p$).
\item Substitute the result gotten above into the sum over $k$ and use Wolfram Alpha again to simplify.
\end{enumerate}
\end{rem}

\subsubsection*{The last sum in (\ref{eqn:partialres})}

We will first try to simplify the very last sum in (\ref{eqn:partialres}). Using (\ref{eqn:Dskipfactsimp}), we can rewrite the last sum in (\ref{eqn:partialres}) as the following:
\begin{align*}
&V_{1}\sum\limits_{\substack{1\leq p \leq m-1 \\ 0 \leq q \leq m - 1 - p}}\frac{1}{2p+1}\prod\limits_{k=p+1}^{p+q}\left(\frac{2k}{2k+1}\right) \\
&\qquad= V_1\sum\limits_{\substack{1\leq p \leq m-1 \\ 0 \leq q \leq m-1-p}} \frac{1}{2p+1}\frac{2^{2(p+q)}}{2^{2(p+1)-1}}\frac{(p+1)(p+2)C_{p+1}}{(2(p+q)+1)(p+q+1)C_{p+q}}.
\end{align*}
Setting $k = p+q$, this last sum turns into
\begin{equation}
\begin{aligned}\label{eqn:lastvarchange}
&V_1\sum\limits_{k=1}^{m-1}\sum\limits_{p=1}^k\frac{1}{2p+1}\frac{2^{2k}}{2^{2(p+1)-1}}\frac{(p+1)(p+2)C_{p+1}}{(2k+1)(k+1)C_k} \\
&\qquad= V_1\sum\limits_{k=1}^{m-1}\frac{2^{2k}}{(2k+1)(k+1)C_k}\sum\limits_{p=1}^k\frac{(p+1)(p+2)C_{p+1}}{2^{2p+1}(2p+1)}.
\end{aligned}
\end{equation}

The inner sum on the right hand side of the above can be further simplified
\begin{equation}\label{eqn:lastinnersum}
\begin{aligned}
\sum\limits_{p=1}^k\frac{(p+1)(p+2)C_{p+1}}{2^{2p+1}(2p+1)} &= \sum\limits_{p=1}^k\frac{(p+1)(p+2)\frac{1}{p+2}\binom{2(p+1)}{p+1}}{2^{2p+1}(2p+1)} \\
&= \sum\limits_{p=1}^k\frac{(p+1)\frac{(2(p+1))!}{[(p+1)!]^2}}{2^{2p+1}(2p+1)} \\
&= \sum\limits_{p=1}^k\frac{(2(p+1))!}{p!(p+1)!2^{2p+1}(2p+1)}.
\end{aligned}
\end{equation}

Wolfram Alpha says that the last sum of (\ref{eqn:lastinnersum}) can be simplified to
\begin{equation}\label{eqn:lastinnersum2}
\sum\limits_{p=1}^k\frac{(2(p+1))!}{p!(p+1)!2^{2p+1}(2p+1)} = \frac{(k+1)(2(k+2))!}{2^{2(k+1)}(2k+3)(k+1)!(k+2)!}-1.
\end{equation}

So substituting the right hand side of (\ref{eqn:lastinnersum2}) into (\ref{eqn:lastvarchange}) above we have
\begin{equation}\label{eqn:simplifiedlast2}
\begin{aligned}
&V_1\sum\limits_{k=1}^{m-1}\frac{2^{2k}}{(2k+1)(k+1)C_k}\sum\limits_{p=1}^k\frac{(p+1)(p+2)C_{p+1}}{2^{2p+1}(2p+1)} \\
&\quad=V_1\sum\limits_{k=1}^{m-1}\frac{2^{2k}}{(2k+1)(k+1)C_k}\left[\frac{(k+1)(2(k+2))!}{2^{2(k+1)}(2k+3)(k+1)!(k+2)!}-1\right] \\
&= V_1\left[\sum\limits_{k=1}^{m-1}\frac{2^{2k}}{(2k+1)(k+1)C_k}\cdot\frac{(k+1)(2(k+2))!}{2^{2(k+1)}(2k+3)(k+1)!(k+2)!}\right. \\
&\qquad - \left.\sum\limits_{k=1}^{m-1}\frac{2^{2k}}{(2k+1)(k+1)C_k}\right]
\end{aligned}
\end{equation}

Plugging in the definition of $C_k$, the second sum in (\ref{eqn:simplifiedlast2}) simplifies to:
\begin{equation}\label{eqn:lastsecond}
\begin{aligned}
\sum\limits_{k=1}^{m-1}\frac{2^{2k}}{(2k+1)(k+1)C_k} &= \sum\limits_{k=1}^{m-1}\frac{2^{2k}(k+1)[k!]^2}{(2k+1)(k+1)(2k)!} \\
&= \sum\limits_{k=1}^{m-1}\frac{2^{2k}[k!]^2}{(2k+1)(2k)!} \\
&= \frac{2^{2m}\sqrt{\pi}\Gamma(m+1)}{\Gamma(m+\frac{1}{2})}-2.
\end{aligned}
\end{equation}
where the last equality is given by Wolfram Alpha.

The first sum in (\ref{eqn:simplifiedlast2}) simplifies to:
\begin{equation}\label{eqn:lastfirst}
\begin{aligned}
&\sum\limits_{k=1}^{m-1}\frac{2^{2k}}{(2k+1)(k+1)C_k}\frac{(k+1)(2(k+2))!}{2^{2(k+1)}(2k+3)(k+1)!(k+2)!} \\
&\quad= \sum\limits_{k=1}^{m-1}\frac{2^{2k}}{2^{2(k+1)}}\frac{(k+1)^2(2(k+2))![k!]^2}{(2k+1)(k+1)(2k+3)(k+1)!(k+2)!(2k)!} \\
&\quad= \sum\limits_{k=1}^{m-1}\frac{1}{2^2}\frac{(k+1)(2(k+2))![k!]^2}{(2k+1)(2k+3)(k+1)!(k+2)!(2k)!} \\
&\quad= \sum\limits_{k=1}^{m-1}\frac{1}{2^2}\frac{(2(k+2))![k!]^2}{(2k+1)(2k+3)k!(k+2)!(2k)!} \\
&\quad= \sum\limits_{k=1}^{m-1}\frac{1}{2^2}\frac{(2(k+2))!k!}{(2k+1)(2k+3)(k+2)!(2k)!} \\
&\quad = \sum\limits_{k=1}^{m-1}\frac{1}{2^2}\frac{(2(k+2))!}{(2k+1)(2k+3)(k+1)(k+2)(2k)!} \\
&\quad = \sum\limits_{k=1}^{m-1}\frac{1}{2^2}\frac{(2k+1)(2k+2)(2k+3)(2k+4)}{(2k+1)(2k+3)(k+1)(k+2)} \\
&\quad = \sum\limits_{k=1}^{m-1}\frac{1}{2^2}\frac{(2k+2)(2k+4)}{(k+1)(k+2)} \\
&\quad = \sum\limits_{k=1}^{m-1}\frac{1}{2^2}\frac{2(k+1)2(k+2)}{(k+1)(k+2)} \\
&\quad = m-1.
\end{aligned}
\end{equation}

So putting (\ref{eqn:lastfirst}) and (\ref{eqn:lastsecond}) together we have
\begin{equation}\label{eqn:lastsimp}
\begin{aligned}
V_1\sum\limits_{\substack{1\leq p \leq m -1 \\ 0 \leq q \leq m-1-p}}\frac{1}{2p+1}\prod\limits_{k=p+1}^{p+q}\left(\frac{2k}{2k+1}\right) = V_1\left(m+1 - \frac{\sqrt{\pi}\Gamma(m+1)}{\Gamma(m+\frac{1}{2})}\right).
\end{aligned}
\end{equation}

\subsubsection*{The second and third sum in (\ref{eqn:partialres})}

We employ a similar process to simplify the other double sums that appear in (\ref{eqn:partialres}). We first simplify
\begin{equation}\label{eqn:third}
6\sum\limits_{\substack{2 \leq p \leq m \\ 0 \leq q \leq m - p}} \frac{1}{2p-1}\left(\prod\limits_{k=p+1}^{p+q}\frac{(2k-2)}{(2k-1)}\right)\left(\frac{1}{2m+1}\right).
\end{equation}

Here, we've dropped the subscripts on the variables $p_1$ and $q_1$ for convenience. We can factor out $\frac{1}{2m+1}$ first, so we'll disregard it, along with the 6 on the outside, in our calculations below. Using (\ref{eqn:Nskipfactsimp}), we have that (\ref{eqn:third}) (without the $\frac{1}{2m+1}$ term) is equal to
\begin{equation}\label{eqn:thirdsimplified1}
\sum\limits_{\substack{2\leq p \leq m \\ 0 \leq q \leq m-p}} \frac{1}{2p-1}\frac{2^{2(p+q)-1}}{2^{2p-1}}\frac{p(p+1)C_p}{(p+q)(p+q+1)C_{p+q}}.
\end{equation}

Setting $k = p + q$, (\ref{eqn:thirdsimplified1}) becomes
\begin{equation}\label{eqn:thirdvarchange}
\begin{aligned}
&\sum\limits_{\substack{2\leq p \leq m \\ 0 \leq q \leq m-p}} \frac{1}{2p-1}\frac{2^{2(p+q)-1}}{2^{2p-1}}\frac{p(p+1)C_p}{(p+q)(p+q+1)C_{p+q}} \\
&\qquad= \sum\limits_{k=2}^m\sum\limits_{p=2}^k \frac{1}{2p-1}\frac{2^{2k-1}}{2^{2p-1}}\frac{p(p+1)C_p}{k(k+1)C_k} \\
&\qquad= \sum\limits_{k=2}^m\frac{2^{2k-1}}{k(k+1)C_k}\sum\limits_{p=2}^k\frac{p(p+1)C_p}{2^{2p-1}(2p-1)}
\end{aligned}
\end{equation}

Plugging in the definition of $C_p$, the sum over $p$ in (\ref{eqn:thirdvarchange}) simplifies to
\begin{equation}\label{eqn:thirdinnersum}
\begin{aligned}
\sum\limits_{p=2}^k\frac{p(p+1)C_p}{2^{2p-1}(2p-1)} &= \sum\limits_{p=2}^k\frac{p(p+1)\frac{1}{p+1}\frac{(2p)!}{[p!]^2}}{2^{2p-1}(2p-1)} \\
&= \sum\limits_{p=2}^k\frac{p(2p)!}{2^{2p-1}(2p-1)[p!]^2} \\
&= \sum\limits_{p=2}^k\frac{(2p)!}{2^{2p-1}(2p-1)p!(p-1)!} \\
&= \frac{k(2(k+1))!}{2^{2k}(2k+1)k!(k+1)!}-1,
\end{aligned}
\end{equation}

where the last equality is given by Wolfram Alpha. Simplifying the sum over $k$ in (\ref{eqn:thirdvarchange}), we have
\begin{equation}\label{eqn:thirdoutersum}
\begin{aligned}
\sum\limits_{k=2}^m\frac{2^{2k-1}}{k(k+1)C_k} &= \sum\limits_{k=2}^m\frac{2^{2k-1}}{k(k+1)\frac{1}{k+1}\frac{(2k)!}{[k!]^2}} \\
&= \sum\limits_{k=2}^m \frac{2^{2k-1}[k!]^2}{k(2k)!} \\
&= \frac{\sqrt{\pi}\Gamma(m+1)}{\Gamma(m+\frac{1}{2})}-2,
\end{aligned}
\end{equation}
where the last equality is given by Wolfram Alpha (or by recognizing that it's the same sum as (\ref{eqn:lastsecond}) above).

Then plugging in (\ref{eqn:thirdinnersum}) and (\ref{eqn:thirdoutersum}) into (\ref{eqn:thirdvarchange}) above, we have

\begin{equation}
\begin{aligned}
&\sum\limits_{k=2}^m\frac{2^{2k-1}}{k(k+1)C_k}\sum\limits_{p=2}^k\frac{p(p+1)C_p}{2^{2p-1}(2p-1)} \\
&\qquad = \sum\limits_{k=2}^m\frac{2^{2k-1}[k!]^2}{k(2k)!}\frac{k(2(k+1))!}{2^{2k}(2k+1)k!(k+1)!} - \left[\frac{\sqrt{\pi}\Gamma(m+1)}{\Gamma(m+\frac{1}{2})}-2\right] \\
&\qquad = \sum\limits_{k=2}^m\frac{2^{2k-1}}{2^{2k}}\frac{[k!]^2k(2(k+1))!}{k(2k)!(2k+1)k!(k+1)!} - \frac{\sqrt{\pi}\Gamma(m+1)}{\Gamma(m+\frac{1}{2})} + 2 \\
&\qquad = \sum\limits_{k=2}^m \frac{1}{2}\frac{k!}{(k+1)!}\frac{(2k+1)(2k+2)}{(2k+1)} -\frac{\sqrt{\pi}\Gamma(m+1)}{\Gamma(m+\frac{1}{2})} + 2 \\
&\qquad = \sum\limits_{k=2}^m \frac{1}{2}\frac{(2k+2)}{(k+1)} - \frac{\sqrt{\pi}\Gamma(m+1)}{\Gamma(m+\frac{1}{2})} + 2 \\
&\qquad = \sum\limits_{k=2}^m \frac{1}{2}\frac{2(k+1)}{(k+1)}  - \frac{\sqrt{\pi}\Gamma(m+1)}{\Gamma(m+\frac{1}{2})} + 2 \\
&\qquad = m - 1 - \frac{\sqrt{\pi}\Gamma(m+1)}{\Gamma(m+\frac{1}{2})} + 2 \\
&\qquad = m + 1 - \frac{\sqrt{\pi}\Gamma(m+1)}{\Gamma(m+\frac{1}{2})}.
\end{aligned}
\end{equation}

So we have that 
\begin{equation}\label{eqn:thirdsimp}
\begin{aligned}
6\sum\limits_{\substack{2 \leq p \leq m \\ 0 \leq q \leq m - p}} \frac{1}{2p-1}\left(\prod\limits_{k=p+1}^{p+q}\frac{(2k-2)}{(2k-1)}\right)\left(\frac{1}{2m+1}\right) &= 6\frac{1}{2m+1}\left(m + 1 -\frac{\sqrt{\pi}\Gamma(m+1)}{\Gamma(m+\frac{1}{2})}\right).
\end{aligned}
\end{equation}

Applying the exact same process to the second sum in (\ref{eqn:partialres}), we have
\begin{equation}\label{eqn:secondsimp}
\begin{aligned}
&6\sum\limits_{\substack{2\leq p \leq m -1 \\ 0 \leq q \leq m - 1 - p}}\frac{1}{2p-1}\left(\prod\limits_{k=p+1}^{p+q}\frac{(2k-2)}{(2k-1)}\right)\left(\frac{2m}{(2m-1)(2m+1)}\right) \\
&\qquad = 6\left(\frac{m}{(2m-1)(2m+1)}\right)\left(m-\frac{\sqrt{\pi}\Gamma(m)}{\Gamma(m-\frac{1}{2})}\right).
\end{aligned}
\end{equation}

\pagebreak

\bibliographystyle{alpha}
\bibliography{refs.bib}

\end{document}